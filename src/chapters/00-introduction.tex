\chapter*{Úvod}
\addcontentsline{toc}{chapter}{Úvod}
\markboth{Úvod}{Úvod}

\section{Motivácia}

V poslednej dobe sa do popredia svetových a regionálnych udalostí dostali mnohé kybernetické útoky. Prakticky hneď začiatkom roka 2025 sa
Slovensko stalo cieľom tzv. kybernetického útoku, kedy hackerská skupina zaútočila na informačné systémy Úradu geodézie, kartografie a katastra SR
(ÚGKK)\cite{utok_na_kataster_aktuality}\footnote{Konkrétny deň útoku nebol jasne určený, oznam na portáli ÚGKK bol zverejnený 7. januára, no sú náznaky,
že útok prebehol už v nedeľu 5. januára, kedy už bol systém nedostupný.}. Tento útok vyradil štátnu elektronickú službu katastra, pričom hackeri zašifrovali
ostré (produkčné) dáta a údajne si pýtali výkupné v hodnote niekoľkých miliónov dolárov\cite{utok_na_kataster_vykupne}.

\todo{Info z 2023 o audite katastra..?}

Pár dni po útoku zverejnil jeden užívateľ na socialnej sieti LinkedIn príspevok\cite{linkedin_post_mortem_analyza_perimetra_katastra}, v ktorom vykonal
\enquote{\textit{post portem} analýzu perimetra Katastra z verejných zdrojov} pomocou nástroja Shodan. V tejto analýze autor identifikoval mnohé zraniteľnosti
informačného systému --- zastarané verzie protokolu TLS (verzie 1.1 a 1.2), zastarané verzie PHP a Apache Server (z rokov 2009 a 2010), otvorené porty
(123 --- NTP a 161 --- SNMP)\footnote{Pomocou útoku na otvorený port protokolu NTP možno zneužiť na amplifikovanie útoku DDoS.} alebo 230 softvérový
zraniteľností\footnote{10 z nich bolo CVE a mnohé z nich umožňovali vykonávanie ľubovoľného kódu.}. V jednoduchosti, mimo iných problémov,
bolo zanedbané patchovanie služieb a systémov, čo vytváralo potenciálny uhol útoku pre útočníka\footnote{Spomenieme ešte, že autor tiež konštatuje,
prečo tieto zraniteľnosti neodhalil systém Achilles vládneho CSIRT a prečo neboli nahlásené jeho zistenia.}.

Obnova funkcionality katastra nebola kompletná ani ešte začiatkom februára\cite{otvaranie_katastralnych_udajov_zive}, pričom nebolo jasné, či má štát dostupné
kompletné zálohy dát na obnovu systému. Čo však vcelku zrejmé bolo, že štát nemal dostupné komplexné zálohy na rýchlu obnovu systému. Nedostupnosť tohto systému vedie
k mnohým komplikáciam života občanov alebo k ekonomickým strátam.

Kybernetické útoky nie sú v súčastnosti ničím výnimočným a dochádza k nim prakticky každý deň (pričom drvivá väčšina z nich je neúspešná). Nedávno došlo k
útoku aj na ďalšie inštitúcie Slovenska (Všeobecnú zdravotnú poisťovňu\cite{utok_na_vszp_zive} alebo Trenčianské vodárne a kanalizácie\cite{utok_na_trencin_zive}).
Samozrejme, Slovensko nie je žiadnou výnimkou, podobne boli zasiahnutý aj naši severní susedia Poliaci, kde došlo k útoku na spoločnosť
EuroCert\cite{utok_na_polsko_zive}\cite{utok_na_polsko_eurocert}\footnote{Unikli citlivé údaje, ako napríklad údaje z občianskych preukazov, čísla PESEL
(poľská verzia rodných čísel), kontaktné údaje, prihlasovacie mená a heslá.}.

Z pohľadu čísel jeden zdroj uvádza \cite{pocet_kyberutokov_na_slovensku}, že v druhej polovici roka 2024 každá organizácia na Slovensku čelila v priemere
1443 kybernetickým útokom každý týždeň, pričom toto číslo stúplo až na 2000 v priebehu decembra\footnote{Najväčší podiel tvorili DoS útoky botnetov s
podielom 11.5\%.}. 

Keby pôjdeme o kúsok ďalej a pozrieme sa na Európu, ENISA vydala v septembri 2024 ročnú správu o stave hrozieb v prostredí kybernetickej
bezpečnosti\cite{enisa_cybersecurity_threat_landscape_2024}. V tejto správe pozorovali 11,709 incidentov a identifikovala niekoľko primárnych hrozieb ---
ransomvér, malvér, sociálne inžinierstvo, hrozba voči dátam, DoS útoky a manipuláciu informácií. Útoky boli najviac zamerané na verejný sektor a tvorili
podiel až 19\%(1870 incidentov), najpočetnejšiu skupinu hrozieb tvorili útoky DOS/DDOS/RDOS (46.31\%, 2460 incidentov), ransomvér (27.33\%, 1450 incidentov)
a útok na dáta\footnote{V zmysle úmyselné narušenie údajov alebo neúmyselný únik údajov.} (18.57\%, 840 incidentov).

Týmto krátkym úvodom sme chceli čitateľovi naznačiť a upevniť v tom, že súčasné informačné systémy napojené na internet sú vystavené veľkému počtu hrozieb.
Chrániť svoje systému musí každá organizácia, či je malá alebo veľká, pretože potenciálne (v niektorých prípadoch aj reálne) negatívne následky často
rádovo prevyšujú náklady, ktoré by mohli organizácie vynaložiť na preventívne opatrenia a systematické riešenie informačnej a kybernetickej bezpečnosti.
Nehovoriac o Slovenskej legislatíve, ktorá ukladá veľkému množstvu subjektov mnohé zákonné povinnosti v oblasti informačnej a kybernetickej bezpečnosti.
Preto položíme čitateľovi otázku v kontexte predchádzajúcich odstavcov: \enquote{Keď už aj veľké inštitúcie ako ÚGKK majú problém so zaistením dostatočnej
informačnej a kybernetickej bezpečnosti, o čo lepšie na tom budú malé samosprávy s menšími kapacitami a zdrojmi?}

%todo a Markízu\cite{it_expert_markiza},
\todo{vieme nájsť aj citáciu záberu z markízy?}

Jeden IT expert uviedol pre Rádio Expres\cite{it_expert_radio_express}, že v prepočte na počet obcí nám na Slovensku chýba
zhruba 20,000 expertov v oblasti kybernetickej a informačnej bezpečnosti. My si tento problém uvedomujeme a preto by sme aj chceli navrhnúť systematické
riešenie, ktorým by sme ho vyriešili. Čím sa dostávame k našej diplomovej práci: \enquote{Čo keby existoval expertný systém pre malé samosprávy, ktorý by
vedel laikovi pomôcť vysvetliť kybernetickú a informačnú bezpečnosť, identifikovať aktíva, posúdiť potenciálne hrozby a prijať bezpečnostné opatrenia na
ošetrenie dopadu rizík?} Návrhu a zdrojom potrebným na vypracovanie takéhoto systému sa budeme venovať v ďalších častiach práce.

\section{Legislatívny rámec}

Problematike kybernetickej a informačnej bezpečnosti sa v legislatívnom rámci Slovenskej republiky venujú najmä dva zákony, a to
Zákon o kybernetickej bezpečnosti\cite{sk_69_2018}(ZoKB) a Zákon o informačných technológiách vo verejnej správe\cite{sk_95_2019}(ZoITVS).
Oba zákony vychádzajú z medzinárodných štandardov série ISO/IEC 27000, sú navzájom prepojené a vzájomne na seba odkazujú.

\subsection*{69/2018 Z. z., zákon o kybernetickej bezpečnosti}
Tento zákon patrí pod pôsobnosť Národného bezpečnostného úradu (NBÚ) a definuje základný subjekt --- prevádzkovateľ základnej služby --- to je ten,
kto je zapísaný v registry prevádzkovateľov základnej služby. Register základných prevádzkovateľov služieb vedie a spravuje NBÚ a v súčasnosti je
v ňom zapísaných vyše 1500 subjektov, ako napríklad rôzne obce, mestá, nemocnice a iné.

Analýza rizík sa explicitné spomína v \par20, ods. (1):
\begin{quote}
[\ldots] Bezpečnostné opatrenia sú realizované na základe vykonanej analýzy rizík a s prihliadnutím na bezpečnostné metodiky a politiky úradu, najnovšie bezpečnostné
trendy a medzinárodné normy a v súlade s bezpečnostnými štandardami v oblasti kybernetickej bezpečnosti a prijímajú sa s cieľom [\ldots]
\end{quote}

Na subjekty kladie ZoKB rôzne požiadavky v oblasti KIB a ukladá im povinnosť zaviesť bezpečnostné opatrenia v závislostí od kategórie sietí a informačných
systémov\footnote{Kategórie sú tri v závislosti od rozsahu potencionálneho bezpečnostného incidentu.}. Rozsah a podrobnosti bezpečnostných oprávnení
upravuje samostatná vyhláška 362/2018 Z. z.\cite{sk_362_2018}.

\subsection*{95/2019 Z. z., zákon o informačných technológiách vo verejnej správe}
Tento zákon patrí pod pôsobnosť Ministerstva investícií, regionálneho rozvoja a informatizácie (MIRRI) a definuje nasledovné pojmy a subjekty:
\begin{itemize}
    \item \textbf{informačná technológia}, prostriedok alebo postup slúžiaci na spracúvanie údajov alebo informácií v elktronickej podobe (informačný systém,
    infraštruktúra, informačná činnosť alebo elektronické služby) 
    \item \textbf{informačný systém}, funkčný celok zabezpečujúci cieľavedomú informačnú a systematickú činnosť prostredníctvom technických a programových prostriedkov
    \item \textbf{správca}, orgán riadenia zodpovedný za informačnú technológiu na účely poskytovania služby verejnej správy
    \item \textbf{prevádzkovateľ}, osobitným predpisom ustanovený orgán riadenia alebo správcom určená osoba a vykonáva činnosti, ktoré mu určí správca\footnote{V skrate,
    správca môže delegovať svoje činnosti, no stále za ňu nesie zodpovednosť.}
\end{itemize}
Rámcovo definuje kompetencie, právomoci a povinnosti správcu pre oblasť KIB, ako napríklad povinnosti pri navrhovaní, akvizícii a obstarávaní informačných systémov alebo
povinnosti voči tretím stranám. Podobne, ako pri ZoKB, kategorizácia sietí a informačných systémov je podrobne rozpracovaná vo vyhláške 179/2020 Z. z.\cite{sk_179_2020}.

Analýza rizík sa explicitné spomína v \par14, ods. (1), písm. h):
\begin{quote}
(1) Správca je na úseku plánovania a organizácie informačných technológií verejnej správy povinný 

\hspace{3em}h) zabezpečiť riadenie rizík,
\end{quote}

\section{Základné pojmy}

Štandardom v oblasti kybernetickej a informačnej bezpečnosti sa venujú viaceré veľké organizácie, terminológia ešte nie je jednotná, ako to uvádza ENISA
vo svojom prehľade o medzerách a prienikoch v štandardizácií\cite{enisa_cybersecurity_definition_gaps}. Preto sa v tejto časti budeme venovať terminológii,
aby sme mali rovnakú predstavu a pochopenie za jednotlivými pojmami, vychádzať budeme z materiálu \enquote{Krátky úvodu do informačnej a kybernetickej
bezpečnosti}\cite{KUdIKBaMVS}.

\subsection*{informácia}
Esenciálny zdroj, bez ktorého organizácia nemôže plniť svoje poslanie. Spracováva sa digitálnymi informačnými a komunikačnými technológiami.

\subsection*{informačná bezpečnosť}
Zaoberá sa hrozbami voči informáciám a hľadaním účinných bezpečnostných opatrení na zníženie rizika, ktoré vplýva z naplnenia týchto hrozieb.

\subsection*{informačné a komunikačné technológie, IKT}
Metódy, prostriedky a zariadenia slúžiace na záznam, prenos, uchovávanie a spracovanie informácie.

\subsection*{digitálne informačné a komunikačné technológie, d-IKT}
IKT, ktoré vznikli spojením počítačov, telekomunikačných sieti a masovokomunikačných prostriedkov, ktoré využívajú digitálne kódovanie informácie
a spoločné komunikačné kanály na prenos údajov.

\subsection*{aktívum}
Čokoľvek, čo má pre organizáciu hodnotu. Môže byť hmotné --- zariadenie, personál alebo nehmotné --- informácia, vedomosť.

\subsection*{hrozba}
Objektívne existujúca potenciálna možnosť priamo alebo nepriamo narušiť systém, spracovávanú informáciu alebo iné aktíva organizácie.

\subsection*{riziko}
Veličina, ktorá závisí od závažnosti/možného dopadu hrozby a pravdepodobnosti naplnenia hrozby.

\subsection*{správa rizík}
Proces identifikácie, odhadu,  vyhodnotenia rizík a prójmania opatrení voči nim. Patrí sem tiež monitorovanie zostatkových rizík a
pravideln=e prehodnocovanie rizík.

\subsection*{analýza rizík}
Proces, ktorý identifikuje, posudzuje, hodnotí a ošetruje riziko.

\subsection*{bezpečnostné opatrenie}
Technické, organizačné, právne alebo iné riešenie, ktoré úplne alebo čiastočne odstraňuje zraniteľnosť, znižuje pravdepodobnosť naplnenia hrozby alebo
v prípade naplnenia hrozby znižuje jej dopad.

\subsection*{dôvernosť, confidentiality}
Bezpečnostná požiadavka, ktorej naplnenie znamená, že prístup k informácií obsiahnutej v správe majú iba povolené osoby.

\subsection*{integrita, integrity}
Bezpečnostná požiadavka, ktorej naplnenie znamená, že údaje nie je nemôže zmeniť bez toho, aby to ich vlastník alebo adresát nemohol zistiť.

\subsection*{dostupnosť, availability}
Bezpečnostná požiadavka, ktorej naplnenie znamená, že zdroje systému sú k dispozícií oprávnenej osobe vždy alebo od času t, keď o to požiada.

\subsection*{CIA}
Anglická skratka iniciálok trojice bezpečnostných požiadaviek --- C(onfidentiality), I(ntegrity), A(vailability).

\section{Organizácie a štandardy venujúce sa analýze rizík}

Pri návrhu a vývoji nášho systému budeme vychádzať z medzinárodných štandardov a metodík. Dôvodom je záruka kvalitných podkladov a vedomostí expertov,
ktorý majú mnohé praktické skúsenosti z problematikou KIB a súčasne kompatibilita ich riešení s legislatívnymi požiadavkami.

\subsection*{NIST}

NIST alebo \textit{angl.} \enquote{National Institute of Standards and Technology} je organizácia Spojených štátov amerických, ktorej cieľom je podpora inovácií
a priemyselnej schopnosti pokrokmi v oblasti vedeckých meraní, štandardov a technológií spôsobmi, ktoré obohacujú ekonomickú bezpečnosť a kvalitu
našich\footnote{Naších v zmysle občanov Spojených štátov amerických} životov\cite{nist_about}. NIST vypracoval viacero štandardov v oblasti správy rizík,
napríklad \textit{NIST Special Publication 800-37}\cite{nist_sp_800_37} a \textit{NIST Special Publication 800-30}\cite{nist_sp_800_30}.

\textbf{NIST Special Publication 800-37} \textit{Risk Management Framework for Information Systems and Organizations} poskytuje usmernenie pre implementáciu
štruktúrovaného a znovupoužiteľný proces riadenia rizík. Začleňuje prvky bezpečnosti, súkromnosti a riadenia rizík dodávateľského reťazca. Implementácia pozostáva
zo siedmych krokov:

\begin{enumerate}
    \item \textbf{príprava} esenciálnych aktivít potrebných pre riadenie bezpečnosť a ochranu súkromia v organizácií --- vymedzenie rolí a zodpovedností,
    identifikácia tolerancie k risku a bezpečnostných požiadaviek a ďalšie
    \item \textbf{kategorizácia} systémov a spracovávanej, uchovávanej a prenášanej informácie voči požiadavkám CIA
    \item \textbf{voľba} bezpečnostných opatrení zo štandardu NIST SP 800-53\cite{nist_sp_800_53} na ochranu systému vychádzajúce z posúdení rizík
    \item \textbf{implementácia} bezpečnostných opatrení a dokumentácia ich nasadenia
    \item \textbf{posúdenie} súladu zavedených bezpečnostných opatrení, či sú funkčné a prinášajú požadované výsledky
    \item \textbf{autorizácia} bezpečnostných opatrení vedením organizácie a určenie, či úroveň rizika systém je akceptovateľná
    \item \textbf{monitorovanie} bezpečnostných opatrení pomocou logov, skenovania zraniteľností, hodnotenia bezpečnostných opatrení a prispôsobovanie
    sa novým hrozbám
\end{enumerate}

\textbf{NIST Special Publication 800-30} \textit{Guide for Conducting Risk Assessments} poskytuje návod, ktorý ma pomôcť organizáciám identifikovať a
posúdiť bezpečnostné riziko ich systémov a dát za použitia stratégie \textit{risk-based decision-making}\footnote{Myšlienkou \textit{risk-based decision-making}
je vyhodnocovať riziká, príležitosti, potenciálne výsledky a voľba cesty k dlhodobému úspechu.}. Implementácia pozostáva zo štyroch krokov:
\begin{enumerate}
    \item \textbf{príprava}, definovanie účelu, rozsahu, predpokladov a zhromaždenie dát pre analýzu rizík
    \item \textbf{vykonanie} analýzy rizík, a to identifikáciou zdrojov hrozieb, zraniteľností a určením pravdepodobnosti a dopadu
    \item \textbf{komunikácia} zistení analýzy rizík zainteresovaným stranám, vedúcim a IT oddeleniam
    \item \textbf{udržiavanie} analýzy rizík, aktualizácia a revidovanie v pravidelných intervaloch
\end{enumerate}
V príručke je každý z týchto krokov ešte rozdelený na podúlohy a obsahuje zoznam kľúčových aktivít pre daný krok.

\subsection*{ISO/IEC 27000}
ISO \textit{angl.} \enquote{International Organization for Standardization} je medzinárodne uznávaná organizácia publikujúca svetové štandardy, v oblasti
kybernetickej a informačnej bezpečnosti je pre nás zaujímavá séria ISO/IEC 27000\cite{iso_27000}. Na základe tejto série sa môže organizácia certifikovať
ISO 27001 certifikáciou. Mnohé ďalšie štandardy a príručky sa odkazujú na túto sériu a ich kompatibilitu s jej štandardami. Nás z tejto série bude zaujímať
najmä štandard ISO/IEC 27005\cite{iso_27005} zameraný na riadenie rizík informačnej bezpečnosti.

Štandard \textbf{ISO/IEC 27005} poskytuje štrukturovaný prístup k identifikácií, posudzovaniu a ošetrení rizík kybernetickej a informačnej bezpečnosti.
Implementácia pozostáva z piatich krokov:
\begin{enumerate}
    \item \textbf{určenie kontextu}, definovanie rozsahu, cieľa, kritérií rizika, aktív, zainteresovaných strán, biznisových procesov a určenie akceptovateľného rizika
    \item \textbf{posudzovanie rizika}, identifikácia (hrozieb, zraniteľností a potenciálnych dopadov), analýza (určenie pravdepodobnosti a dopadu každého rizika)
    a vyhodnotenie (porovnanie voči akceptovateľnému riziku) rizík 
    \item \textbf{ošetrenie rizík}, prijatie bezpečnostných opatrení na zmiernenie, prenos, akceptovanie alebo prenesenie rizika
    \item \textbf{komunikácia rizík}, zistení analýzy rizík zainteresovaným stranám, vedúcim a IT oddeleniam a zaručenie súladu s biznisovými cieľmi organizácie
    \item \textbf{monitorovanie a kontrola}, kontinuálne monitorovanie a aktualizácia rizík, vykonanie bezpečnostného auditu a penetračného testovania
\end{enumerate}

\subsection*{BSI}
BSI \textit{nem.} \enquote{Bundesamt für Sicherheit in der Informationstechnik} je nemecký spolkový úrad pre informačnú bezpečnosť. BSI vydalo niekoľko štandardov
v oblasti kybernetickej a informačnej bezpečnosti, konkrétne BSI-Standard 200-3\cite{bsi_200_3} a IT-Grundschutz-Compendium\cite{bsi_it_grundchutz_compendium}.
Dôležité je tiež spomenúť, źe tieto štandardy sú kompatibilné so sériou ISO/IEC 27000.

\textbf{IT-Grundschutz-Compendium} poskytuje štandardizované požiadavky pre typické biznisové procesy (napr. OPS.1.1.3 Patch and Change Management),
aplikácie (napr. APP.1.1 Office Products), IT systémy (napr. SYS.1.3 Linux and Unix Servers ), komunikačné linky (napr. NET.1.1 Network Architecture and Design)
a miestnosti (napr. INF.2 Data Centre and Server Room) popísané v samostatných moduloch IT-Grundschutz.

Hádam najväčiou výhodou IT-Grundschutz-Compendium je zníženie analytického úsilia na organizácie. BSI pre jednotlivé moduly už vykonala  analýzu rizík,
hrozieb a zraniteľností a vybralo vhodné bezpečnotné požiadavky pre typické scenáre, ktoré si môže organizácia prebrať a implementovať do vlastných 
bezpečnostných opatrení v závislosti od ich vlastnej situácie, potrieb a zdrojov.

\textbf{BSI-Standard 200-3} \textit{Risk Analysis based on IT Grundschutz} poskytuje jednoducho aplikovateľné a rozoznateľné procedúry, ktoré organizaciám
umožňujú adekvátnu cielenú kontrolu rizík kybernetickej a informačnej bezpečnosti. Je to praktický návod pre analýzu rizík založený na IT-Grundschutz-Compendium
opisujúci postupnosť krokov s príkladmi a vysvetleniami. Implementácia pozostáva zo štyroch krokov:
\begin{enumerate}
    \item \textbf{vypracovanie prehľadu hrozieb}, zostavenie zoznamu potenciálnych základných hrozieb, určenie dodatočných hrozieb vyplývajúcich zo
    špecifických operačných scenárov
    \item \textbf{klasifikácia rizík}, posudzovanie (určenie pravdepodobnosti a rozsahu škôd) a hodnotenie (určenie kategórie) rizík
    \item \textbf{ošetrenie rizík}, vyhnutie sa riziku, zníženie rizika (určenie bezpečnostných opatrení), prenos rizika a akceptovanie rizika
    \item \textbf{konsolidácia bezpečnostného konceptu}, integrovanie dodatočných bezpečnostných opatrení vyplývajúcich z analýzy rizík v bezpečnostnom koncepte
\end{enumerate}

\section{Výber štandardov pre náš systém}

Jednou z výhod štandardov NIST je ich rozsiahlosť a obsažnosť, na druhú stranu ich cieľom sú veľké podniky na pôde Spojených štátov amerických. Nevylučujeme však,
že pre manažéra kybernetickej a informačnej bezpečnosti by boli cenným zdrojom znalostí.

Podobne, séria štandardov ISO/IEC 27000 rozsiahle opisuje požiadavky a spôsob vykonania analýzy rizík. Negatívum je, že nevysvetľujú postup tak \enquote{ľudsky},
ako štandardy BSI, nie sú verejne dostupné a ani nie sú cenovo najdostupnejšie.

Nakoniec sme sa rozhodli postupovať a vychádzať podľa štandardov BSI. Pre návrh a implementáciu nášho systému sme sa rozhodli vychádzať zo štandardu
BSI-Standard 200-3\cite{bsi_200_3} a kompendia IT-Grundschutz-Compendium\cite{bsi_it_grundchutz_compendium}. Medzi ich hlavné výhody považujeme:
\begin{itemize}
    \item praktické návody na zavedenie kybernetickej a informačnej bezpečnosti
    \item vypracovanie analýzy rizík pre štandardné situácie
    \item modulárny prístup
    \item verejná dostupnosť
    \item dobre aplikovateľné v priestore Európskej únie a Slovenskej republiky
    \item dajú sa dobre prispôsobiť rôzne veľkým/komplexným organizáciám
    \item kompatibilita so série štandardami ISO/IEC 27000
\end{itemize}

Legislatíva SR v oblasti KIB vychádzala zo série štandardov ISO/IEC 27000, teda aplikovaním postupov a bezpečnostných opatrení štandardov BSI by sme dosiahli
súlad aj s našou legislatívou.


\section{Požiadavky na systém}

Náš systém bude postavený na štandarde BSI-Standard 200-3\cite{bsi_200_3} a moduloch kompendia IT-Grundschutz-Compendium\cite{bsi_it_grundchutz_compendium}.
Používateľovi by mal pomôcť nasledovne:
\begin{enumerate}
    \item oboznámi ho o jeho zákonných povinnostiach
    \item poskytne mu zoznam aktív pre analýzu rizík, z ktorých si zvolí relevantné aktíva
    \item prevedie ho postupom analýzy rizík pre konkrétne aktívum
    \item navrhne mu bezpečnostné opatrenia na ošetrenie rizík
    \item vytvorí bezpečnostnú dokumentáciu vykonanej analýzy rizík
\end{enumerate}

Systém bude navrhnutý modulárne, aby poskytoval jednotný prístup pri analýze rizík a aby mohol byť v budúcnosti ľahko rozšíriteľný.
Cieľová skupina nášho systému sú malé samosprávy, ktoré nemajú dostatočné prostriedky alebo kapacity, aby si analýzu rizík sieti a informačných systémov
urobili sami.

Naskytá sa otázka, ako navrhnúť takýto systém, aby bol použiteľný, modulárny a intuitívny pre používateľa? Touto otázkou sa budeme zaoberať v ďalších kapitolách.